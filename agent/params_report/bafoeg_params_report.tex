\documentclass[a4paper,11pt]{article}
\usepackage[margin=1in]{geometry}
\usepackage[utf8]{inputenc}
\usepackage[T1]{fontenc}
\usepackage[ngerman]{babel}
\usepackage{booktabs}
\usepackage{array}
\usepackage{hyperref}
\usepackage{longtable}
\usepackage{amsmath}
\usepackage{apalike}
\usepackage{graphicx} 
\usepackage{rotating}

\title{Parametrisierung f\"ur die BAf\"oG-OCEL-Simulation \\[1ex] \large Business Process Management (WS 25/26) \\ Group 10}
\author{Alina Thoden \\ Haya Edris \\ Mansour Dukhan \\ David Derr \\ Büsra Bugrahan \\ Jan Gödicke}
\date{\today}

\begin{document}
\maketitle

\section{Prozesskontext}

\subsection{Szenario und Datengrundlage}
Gegenstand der Simulation sind Eingänge von BAföG-Anträgen für Hamburg zum Wintersemester 2024/2025. Die Spitzenlast wird durch den Semesterstart geprägt, wobei etwa 63\,\% der Anträge um den Winterstart eingehen~\cite{sh2024}. Als Datengrundlage für Hamburg werden 15.564 geförderte Studierende im Jahr 2024 zugrunde gelegt~\cite{destatis2024}.

\subsection{Prozessfokus}
Der Fokus liegt auf der Bearbeitung von Erstanträgen, die ausschließlich digital über BAföG-Digital eingereicht werden. Das Szenario betrachtet damit nur Anträge über BAföG-Digital und schließt Papier- oder Mischkanäle sowie Folgeanträge, Schüler- und Auslands-BAföG aus. Eine Trennung zwischen Inlands- und Auslands-BAföG ist möglich, da die Personalkapazität auf diese aufgeteilt wird~\cite{bewilligung2012}.
\\\\
Startpunkt ist „Application started", also der Zeitpunkt des Dateneinggangs im System. Vorarbeiten des Studierenden bleiben unberücksichtigt, da sie die Amtsdurchlaufzeit nicht beeinflussen.
\subsection{Volumenannahmen}
Wir simulieren ein High-Load-Szenario, bei dem das gesamte Antragsvolumen (9.800 bei Start des Wintersemesters) als Erstantr\"age behandelt wird, um die Belastbarkeit des Systems unter maximaler Komplexit\"at zu pr\"ufen.

\subsection{Modellierungsansatz}
Neues BPMN ist st\"arker datengetrieben als die initiale Version \autoref{fig:bafoeg_ocel}.

\begin{sidewaysfigure}
    \centering
    \includegraphics[width=\textwidth]{bafoeg_ocel.png}
    \caption{Aktuelles BPMN des BAföG-Prozesses}
    \label{fig:bafoeg_ocel}
\end{sidewaysfigure}

\section{Interarrival}
Die Lastspitzen orientieren sich am Wintersemester-Start. Nach ~\cite{sh2024} gehen etwa 63\,\% der Anträge rund um den Winterstart ein. Auf Hamburg übertragen entspricht das etwa 9.800 von 15.564 geförderten Studierenden~\cite{destatis2024}. Wir nehmen folgende Interarrival-Zeiten an:

\begin{longtable}{@{}p{2.5cm}p{2cm}p{2.5cm}p{3cm}p{5cm}@{}}
\toprule
Zeitfenster & Wochentage & Verteilung & Parameter & Annahme \\
\midrule
08:00--16:00 & Mo--Fr & Exponential & Mittel 120 min & Hauptgesch\"aftszeit, moderater Strom digitaler Eing\"ange\\
16:00--21:00 & Mo--Fr & Exponential & Mittel 30 min & Feierabend-Peak bei Online-Abgabe. \\
21:00--23:59 & Mo--Fr & Exponential & Mittel 180 min & Sp\"atabgabe, geringere Dichte. \\
Ganztags & Sa--So & Exponential & Mittel 300 min & Wochenende, geringere Interarrival (Online-only). \\
\bottomrule
\end{longtable}

\section{Gateways}
\subsection{Parent Data Required?}
Gemäß Statistik Bayern 2022~\cite{statbay2022} sind etwa 18\% der Geförderten elternunabhängig (13.637 von 74.771). Daraus leiten wir ab, dass in 80\% der Fälle Elternunterlagen benötigt werden, während 20\% elternunabhängig gefördert werden. Dies steuert das Attribut \texttt{Application.is\_parent\_independent} (FALSE = Elternunterlagen nötig).

\subsection{Documents Missing?} 
Nach Fachliteratur sind nur 1--2\% der Papier-Erstanträge vollständig~\cite{einfacher2010}, während bei Weiterförderungen etwa 35\% komplett eingereicht werden~\cite{sh2024}. Da der Fokus ausschließlich auf Erstanträgen liegt, ist das Risiko unvollständiger Unterlagen höher (ca. 98\% bei Papier). Durch die Nutzung des digitalen Assistenten BAföG Digital wird eine Verbesserung angenommen, weshalb wir das Risiko ``Documents Missing'' auf 70\% und ``Complete'' auf 30\% setzen. Dieses Gateway wird durch das Attribut \texttt{Document.status} (``Missing'' vs. ``Received'') gesteuert.

\subsection{Eligibility Decision?}
Nach~\cite{bewilligung2012} liegt der Anteil nicht bewilligter Anträge bei 16\%. Die Pfadwahrscheinlichkeiten betragen somit 84\% für bewilligt und 16\% für abgelehnt. Dieses Gateway wird durch das Attribut \texttt{Application.status} (``Approved'' vs. ``Rejected'') gesteuert.

\section{Aktivit\"atsdauern}
Zur Ableitung der Aktivitätsdauern wurden die Standardzeiten aus Tabelle 22 („Durchschnittliche Standardzeiten der Antragsbearbeitung im Studierendenwerk Hamburg für die Inlandsförderung ")~\cite{bewilligung2012} herangezogen. Für das Szenario Erstantrag ergibt sich daraus eine Gesamtsumme von 83 Minuten. Die Anteile der Einzelaktivitäten wurden proportional zu den in Tabelle 22 angegebenen Standardzeiten bestimmt und auf die nachfolgend aufgeführten Aktivitätsgruppen verteilt:

  \begin{itemize}
      \item \textbf{Receive Application:} enstpricht Anlegen der Papierakte \(\approx 13\,\text{min}\)
      \item \textbf{Review:} entspricht Vollständigkeitsprüfung \(\approx 13\,\text{min}\).
      \item \textbf{Request Missing Data}  entspricht fehlende Daten oder Informationen einholen \(\approx 12\,\text{min}\).
      \item \textbf{Assess:} entspricht Hälfte der Zeit von Berechnungen/Bewertungen durchführen \(\approx 15\,\text{min}\).
      \item \textbf{Calculate:} entspricht Hälfte der Zeit von Berechnungen/Bewertungen durchführen inkl. Ergebnisse prüfen und korrigieren \(\approx 20\,\text{min}\).
      \item \textbf{Notification/Rejection:} entspricht Aufbereiten, Versenden \(\approx 10\,\text{min}\).
  \end{itemize}

Die Aktivitätsdauern der Sachbearbeitung wurden zur Parametrisierung in stochastische Verteilungen überführt. Manuelle Tätigkeiten werden durch Normalverteilungen modelliert (mit plausiblen Standardabweichungen \(\sigma\) sowie angegebenen Minimal- und Maximalwerten), systemseitige Aktivitäten durch Uniformverteilungen und Wartezeiten durch Exponentialverteilungen.
\\
\small
\setlength{\LTpre}{0pt}
\setlength{\LTpost}{0pt}
\begin{longtable}{@{} >{\raggedright\arraybackslash}p{0.18\textwidth}
          >{\raggedright\arraybackslash}p{0.12\textwidth}
          >{\raggedright\arraybackslash}p{0.15\textwidth}
          >{\raggedright\arraybackslash}p{0.22\textwidth}
          >{\raggedright\arraybackslash}p{0.33\textwidth} @{}}
\toprule
Aktivit\"at & Ressource & Verteilung & Parameter & Begr\"undung \\
\midrule
\endhead
Request Parent Data & System & Uniform & 0{,}5--2 min & Automatische Mail an Eltern \\
Receive Parent Data & System & Exponential & Mittel 10{,}080 min, min 1{,}440, max 20{,}160 & Wartezeit auf Elterndaten\\
Send Application Mail & System & Uniform & 1--3 min & Automatisches Generieren + Versand \\
Receive Application & Clerk & Normal & $\mu=13$, $\sigma=4$, min 5, max 25 min & Entspricht Anlegen der Papierakte \\
Review Document & Clerk & Normal & $\mu=13$, $\sigma=4$, min 5, max 25 min & Vollst\"andigkeitspr\"ufung \\
Request Missing Documents & Clerk & Normal & $\mu=12$, $\sigma=3$, min 5, max 25 min & Fehlende Daten einholen \\
Receive Missing Documents & System & Exponential & Mittel 10{,}080 min, min 1{,}440, max 20{,}160 & Wartezeit auf fehlende Dokumente \\
Assess Application & Clerk & Normal & $\mu=15$, $\sigma=4$, min 5, max 30 min & Berechnungen/Bewertungen durchf\"uhren \\
Calculate Claim & Clerk & Normal & $\mu=20$, $\sigma=5$, min 10, max 40 min & Ergebnisse pr\"ufen und korrigieren \\
Send Notification & Clerk & Normal & $\mu=10$, $\sigma=3$, min 5, max 20 min & Bescheid aufbereiten und versenden \\
Send Rejection & Clerk & Normal & $\mu=10$, $\sigma=3$, min 5, max 20 min & Analog Notification. \\
\bottomrule
\end{longtable}
\normalsize

\subsection{Dokumenttypen und Komplexit\"atsfaktoren}
F\"ur eine detaillierte Simulation k\"onnen die Bearbeitungszeiten basierend auf dem Dokumenttyp skaliert werden. Die Faktoren beruhen auf Annahmen.

\begin{center}
\small
\setlength{\tabcolsep}{4pt}
\begin{tabular}{@{} >{\raggedright\arraybackslash}p{0.38\textwidth}
       >{\raggedright\arraybackslash}p{0.30\textwidth}
       >{\raggedright\arraybackslash}p{0.22\textwidth} @{}}
\toprule
Dokument & Bedingung & Komplexitäts\-faktor \\
\midrule
Formblatt 1 (Antrag) & immer & 1.0 \\[2pt]
Immatrikulations\-bescheinigung & immer & 0.5 \\[2pt]
Formblatt 3 (Einkommen Eltern) & falls {\ttfamily Application.\allowbreak is\_parent\_\allowbreak independent = FALSE} & 1.5 \\[2pt]
Einkommens\-nachweise (Eltern) & falls Formblatt 3 vorhanden (i.d.R. 2 Stück) & 1.3 \\[2pt]
Mietbescheinigung (Wohnnachweis) & falls {\ttfamily Application.\allowbreak housing\_type \(\neq\) 'Eltern'} & 0.8 \\
\bottomrule
\end{tabular}
\normalsize
\end{center}

\section{Ressourcen}
\subsection{System}
\begin{itemize}
  \item 24/7 verf\"ugbar, Kapazit\"at hoch (9999)
  \item Verfügbarkeit: 00:00--23:59, alle Tage.
\end{itemize}

\subsection{Clerk (Sachbearbeitung)}
\begin{itemize}
  \item Anzahl: Inlandsförderung mit 32 Sachbearbeitern \cite{bewilligung2012}
  \item Verfügbarkeit: Mo--Fr 07:30--16:00.
\end{itemize}

\section{Geplante Deviations}
Um einen Conformance Check zu demonstrieren, werden gezielt Abweichungen in den Simulationsdaten erzeugt, die vom Standardprozess (Happy Path) abweichen.

\subsection{Switched Activities}
\textbf{Szenario:} Bei 10\,\% der Fälle wird der Schritt \glqq Assess Application\grqq{} mit \glqq Calculate Claim\grqq{} in der Reihenfolge vertauscht. \\\\
\textbf{Bedeutung:} Dies simuliert eine Abweichung der Prozessreihenfolge. Im Conformance Check muss dies als \textbf{Sequence Violation} (Reihenfolgeverletzung) erkannt werden.

\subsection{Direkte Ablehnung (Shortened Path)}
\textbf{Szenario:} Bei 5\,\% der Fälle bricht der Prozess nach \glqq Review Documents\grqq{} (oder direkt bei Eingang) sofort ab und springt zu einem End-Event (z.\,B. \glqq Application Rejected\grqq{}), ohne die weiteren Prüfungsschritte (\glqq Assess\grqq{}, \glqq Calculate\grqq{}) zu durchlaufen. \\\\
\textbf{Bedeutung:} Simuliert eine sofortige Ablehnung (z.\,B. wegen offensichtlicher Formfehler), die im Standardmodell so nicht vorgesehen ist. Im Conformance Check erkennbar als \textbf{Skipped Activities} oder \textbf{unerlaubter Pfad} (fehlende Kante im Modell).

\subsection{Ablehnung wegen Fristablauf}
\textbf{Szenario:} Bei 4\,\% springt der Prozess nach \glqq Request Missing Documents\grqq{} -- nach einer Wartezeit >30 Tage -- direkt zu \glqq Send Rejection\grqq{} (ohne Assess/Calculate). \\\\
\textbf{Bedeutung:} Fehlende Nachreichung führt zu automatischer Ablehnung. Da diese Timeout-Regel nicht im Modell vorgesehen ist, führt dies zu \textbf{Skipped Activities}.

\section{Simulationsempfehlungen}
\begin{itemize}
  \item KPIs: Cycle Time je Antrag, Auslastung Sachbearbeiter, Wartezeiten pro Queue, Bewilligungsquote.
  \item Experimente: Sensitivit\"at auf Dokumentvollst\"andigkeit (Gateway), Variation Receive-Document-Mittelwert, Personalkapazit\"at (z.B. 10--16 FTE), saisonale Peaks vs. Normalsemester.
\end{itemize}

\newpage

\bibliographystyle{apalike}
\begin{thebibliography}{9}

\bibitem[Bundesrechnungshof, 2024]{brh2024}
Bundesrechnungshof (2024):
\textit{Bericht zum BAföG – Volltext}.
Textziffern (Tz.) 2.2.1, 3.2, 9.1, 9.2 u.\,a.
URL: \url{https://www.bundesrechnungshof.de/SharedDocs/Downloads/DE/Berichte/2024/bafoeg-volltext.pdf?__blob=publicationFile&v=2}

\bibitem[Bundesregierung, 2010]{einfacher2010}
Bundesregierung; Nationale Normenkontrollrat (2010):
\textit{Einfacher zum Studierenden-BAföG – Abschlussbericht}.
URL: \url{https://www.bundesregierung.de/resource/blob/2065474/396444/d606b826f7415afb0c091c30e968e383/2010-03-17-abschlussbericht-einfacher-zum-bafoeg-data.pdf?download=1}

\bibitem[Bundesregierung, 2012]{bewilligung2012}
Bundesregierung (2012):
\textit{Projektbericht zur BAföG-Verwaltung – Bewilligungsquoten Inland}.
Tabellen 22 und 23.
URL: \url{https://www.bundesregierung.de/resource/blob/974430/455514/00c68b52fe5801d0f59643345aa942a2/2012-06-22-projektbericht-7-data.pdf?download=1}

\bibitem[Studentenwerk SH, 2024]{sh2024}
Studentenwerk Schleswig-Holstein (2024):
\textit{Warum dauert die BAföG-Bearbeitung länger? – FAQ}.
URL: \url{https://studentenwerk.sh/de/bafoeg-warum-die-antragstellung-laenger-dauert}

\bibitem[Destatis, 2024]{destatis2024}
Statistisches Bundesamt (Destatis) (2024):
\textit{BAföG-Geförderte nach Ländern}.
GENESIS-Tabelle 21411-0020.
URL: \url{https://www-genesis.destatis.de/datenbank/online/statistic/21411/table/21411-0020}

\bibitem[Statistik Bayern, 2022]{statbay2022}
Bayerisches Landesamt für Statistik (2022):
\textit{BAföG-Geförderte – Anteil elternunabhängiger Förderung}.
Statistische Berichte, Kennziffer K9100C.
URL: \url{https://www.statistik.bayern.de/mam/produkte/veroffentlichungen/statistische_berichte/k9100c_202200.pdf}

\end{thebibliography}


\end{document}