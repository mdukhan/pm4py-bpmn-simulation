\documentclass[a4paper,11pt]{article}
\usepackage[margin=1in]{geometry}
\usepackage[utf8]{inputenc}
\usepackage[T1]{fontenc}
\usepackage[ngerman]{babel}
\usepackage{booktabs}
\usepackage{array}
\usepackage{hyperref}
\usepackage{longtable}
\usepackage{amsmath}
\usepackage{apalike}
\usepackage{graphicx} 
\usepackage{rotating}
\usepackage{float}

\title{Parametrisierung für die BAföG-OCEL-Simulation \\[1ex] \large Business Process Management (WS 25/26) \\ Group 10}
\author{Alina Thoden \\ Haya Edris \\ Mansour Dukhan \\ David Derr \\ Büsra Bugrahan \\ Jan Gödicke}
\date{\today}

\begin{document}
\maketitle

\section{Prozesskontext}

\subsection{Szenario und Datengrundlage}
Gegenstand der Simulation sind Eingänge von BAföG-Anträgen für Hamburg zum Wintersemester 2024/2025. Die Spitzenlast wird durch den Semesterstart geprägt, wobei etwa 63\,\% der Anträge um den Winterstart eingehen~\cite{sh2024}. Als Datengrundlage für Hamburg werden 15.564 geförderte Studierende im Jahr 2024 zugrunde gelegt~\cite{destatis2024}.

\subsection{Prozessfokus}
Der Fokus liegt auf der Bearbeitung von Erstanträgen, die ausschließlich digital über BAföG-Digital eingereicht werden. Das Szenario betrachtet damit nur Anträge über BAföG-Digital und schließt Papier- oder Mischkanäle sowie Folgeanträge, Schüler- und Auslands-BAföG aus. Eine Trennung zwischen Inlands- und Auslands-BAföG ist möglich, da die Personalkapazität auf diese aufgeteilt wird~\cite{bewilligung2012}.
\\\\
Startpunkt ist „Application started", also der Zeitpunkt des Dateneinggangs im System. Vorarbeiten des Studierenden bleiben unberücksichtigt, da sie die Amtsdurchlaufzeit nicht beeinflussen.
\subsection{Volumenannahmen}
Wir simulieren ein High-Load-Szenario, bei dem das gesamte Antragsvolumen (9.800 bei Start des Wintersemesters) als Erstanträge behandelt wird, um die Belastbarkeit des Systems unter maximaler Komplexität zu prüfen.

\subsection{Modellierungsansatz}
Das aktuelle BPMN \autoref{fig:bafoeg_ocel} ist stärker datengetrieben als die initiale Version \autoref{fig:bafoeg_v1}. Die aktuelle Version konzentriert sich auf Aktivitäten die Änderungen in den Datenbanken verursachen würden, wie beispielweise CRUD Operationen.

\begin{sidewaysfigure}
    \centering
    \includegraphics[width=\textwidth]{bafoeg_v1.png}
    \caption{Erste Version: BPMN des BAföG-Prozesses}
    \label{fig:bafoeg_v1}
\end{sidewaysfigure}

\begin{sidewaysfigure}
    \centering
    \includegraphics[width=\textwidth]{bafoeg_ocel.png}
    \caption{Aktuelles BPMN des BAföG-Prozesses}
    \label{fig:bafoeg_ocel}
\end{sidewaysfigure}

\subsection{Szenario-Setup \& Output}
\begin{itemize}
    \item \textbf{Zeitraum:} Die Simulation startet am \textbf{15.09.2024} und läuft über einen Zeitraum von \textbf{80 Tagen}, um das gesamte Wintersemester inklusive Nachlaufzeit abzubilden.
    \item \textbf{Reproduzierbarkeit:} Um reproduzierbare Ergebnisse zu gewährleisten, wird ein fester \textbf{Random Seed (42)} verwendet.
    \item \textbf{Output:} Die Simulation generiert relationale \textbf{CSV-Tabellen} (Events, Applications, Documents, Object-Links), die in ihrer Struktur einem Object-Centric Event Log (OCEL) entsprechen. Dies ermöglicht eine direkte Analyse in Process-Mining-Tools, die objekt-zentrierte Daten unterstützen.
\end{itemize}

\section{Interarrival}
Die Lastspitzen orientieren sich am Wintersemester-Start. Nach ~\cite{sh2024} gehen etwa 63\,\% der Anträge rund um den Winterstart ein. Auf Hamburg übertragen entspricht das etwa 9.800 von 15.564 geförderten Studierenden~\cite{destatis2024}. Wir nehmen folgende Interarrival-Zeiten an:

\begin{longtable}{@{}p{2.5cm}p{2cm}p{2.5cm}p{3cm}p{5cm}@{}}
\toprule
Zeitfenster & Wochentage & Verteilung & Parameter & Annahme \\
\midrule
08:00--16:00 & Mo--Fr & Exponential & Mittel 1,5 min & Peak: Hauptgeschäftszeit (High Load Szenario) \\
16:00--21:00 & Mo--Fr & Exponential & Mittel 2,7 min & Peak: Nach Vorlesungen (High Load Szenario) \\
21:00--24:00 & Mo--Fr & Exponential & Mittel 8,1 min & Spätabgabe, moderate Dichte \\
Ganztags & Sa--So & Exponential & Mittel 10,8 min & Wochenende, reduzierte Rate \\
\bottomrule
\\[1ex]
\caption{Interarrival-Zeiten für die Simulation} \label{tab:interarrival} \\
\end{longtable}

\section{Gateways}
\subsection{Parent Data Required?}
Gemäß Statistik Bayern 2022~\cite{statbay2022} sind etwa 18\% der Geförderten elternunabhängig (13.637 von 74.771). Daraus leiten wir ab, dass in 80\% der Fälle Elternunterlagen benötigt werden, während 20\% elternunabhängig gefördert werden. Dies steuert das Attribut \texttt{Application.is\_parent\_independent} (FALSE = Elternunterlagen nötig).

\subsection{Documents Missing?} 
Nach Fachliteratur sind nur 1--2\% der Papier-Erstanträge vollständig~\cite{einfacher2010}, während bei Weiterförderungen etwa 35\% komplett eingereicht werden~\cite{sh2024}. Da der Fokus ausschließlich auf Erstanträgen liegt, ist das Risiko unvollständiger Unterlagen höher (ca. 98\% bei Papier). Durch die Nutzung des digitalen Assistenten BAföG Digital wird eine Verbesserung angenommen, weshalb wir das Risiko ``Documents Missing'' auf 40\% und ``Complete'' auf 60\% setzen. Dieses Gateway wird durch das Attribut \texttt{Document.status} (``Missing'' vs. ``Received'') gesteuert.

\subsection{Eligibility Decision?}
Nach~\cite{bewilligung2012} liegt der Anteil nicht bewilligter Anträge bei 16\%. Die Pfadwahrscheinlichkeiten betragen somit 84\% für bewilligt und 16\% für abgelehnt. Dieses Gateway wird durch das Attribut \texttt{Application.status} (``Approved'' vs. ``Rejected'') gesteuert.

\section{Aktivitätsdauern}
Zur Ableitung der Aktivitätsdauern wurden die Standardzeiten aus Tabelle 22 („Durchschnittliche Standardzeiten der Antragsbearbeitung im Studierendenwerk Hamburg für die Inlandsförderung ")~\cite{bewilligung2012} herangezogen. Für das Szenario Erstantrag ergibt sich daraus eine Gesamtsumme von 83 Minuten. Die Anteile der Einzelaktivitäten wurden proportional zu den in Tabelle 22 angegebenen Standardzeiten bestimmt und auf die nachfolgend aufgeführten Aktivitätsgruppen verteilt:

  \begin{itemize}
      \item \textbf{Receive Application:} entspricht Anlegen der Papierakte \(\approx 13\,\text{min}\)
      \item \textbf{Review:} entspricht Vollständigkeitsprüfung \(\approx 13\,\text{min}\).
      \item \textbf{Request Missing Data}  entspricht fehlende Daten oder Informationen einholen \(\approx 12\,\text{min}\).
      \item \textbf{Assess:} entspricht Hälfte der Zeit von Berechnungen/Bewertungen durchführen \(\approx 15\,\text{min}\).
      \item \textbf{Calculate:} entspricht Hälfte der Zeit von Berechnungen/Bewertungen durchführen inkl. Ergebnisse prüfen und korrigieren \(\approx 20\,\text{min}\).
      \item \textbf{Notification/Rejection:} entspricht Aufbereiten, Versenden \(\approx 10\,\text{min}\).
  \end{itemize}

Die Aktivitätsdauern der Sachbearbeitung wurden zur Parametrisierung in stochastische Verteilungen überführt. Manuelle Tätigkeiten werden durch Normalverteilungen modelliert (mit plausiblen Standardabweichungen \(\sigma\) sowie angegebenen Minimal- und Maximalwerten), systemseitige Aktivitäten durch Uniformverteilungen und Wartezeiten durch Exponentialverteilungen.
\\
\small
\setlength{\LTpre}{0pt}
\setlength{\LTpost}{0pt}
\begin{longtable}{@{} >{\raggedright\arraybackslash}p{0.18\textwidth}
          >{\raggedright\arraybackslash}p{0.12\textwidth}
          >{\raggedright\arraybackslash}p{0.15\textwidth}
          >{\raggedright\arraybackslash}p{0.22\textwidth}
          >{\raggedright\arraybackslash}p{0.33\textwidth} @{}}
\toprule
Aktivität & Ressource & Verteilung & Parameter & Begründung \\
\midrule
\endhead
Application started & System & - & 0 min & Start-Event (Dateneingang) \\
Request Parent Data & System & Uniform & 0{,}5--2 min & Automatische Mail an Eltern \\
Receive Parent Data & External Actor & Normal & $\mu=10080$, $\sigma=3600$ & Wartezeit (Annahme: Normalverteilung) \\
Send Application Mail & System & Uniform & 1--3 min & Automatisches Generieren + Versand \\
Receive Application & Clerk & Normal & $\mu=13$, $\sigma=4$ & Anlegen der Papierakte \\
Review Documents & Clerk & Normal & $\mu=5$, $\sigma=2$ & Zeit pro Dokument (Annahme sim\_config) \\
Request Missing Documents & Clerk & Normal & $\mu=12$, $\sigma=3$ & Fehlende Daten einholen \\
Receive Missing Documents & External Actor & Normal & $\mu=10080$, $\sigma=2880$ & Wartezeit (Annahme: Normalverteilung) \\
Print Documents & Clerk & Uniform & 2--5 min & Medienbruch: Batch printing (Annahme: Deviation) \\
Assess Application & Clerk & Normal & $\mu=15$, $\sigma=4$ & Bewertung \\
Calculate Claim & Clerk & Normal & $\mu=20$, $\sigma=5$ & Berechnung \\
Send Notification & Clerk & Normal & $\mu=10$, $\sigma=3$ & Bescheid Versand \\
Send Rejection & Clerk & Normal & $\mu=10$, $\sigma=3$ & Ablehnung Versand \\
Archive Documents & Clerk & Uniform & 2--5 min & Archivierung (Annahme: Deviation) \\
Application handled & System & - & 0 min & End-Event (Technischer Abschluss) \\
\bottomrule
\\[1ex]
\caption{Verteilung der Aktivitätsdauern} \label{tab:activities} \\
\end{longtable}
\normalsize

\subsection{Dokumenttypen und Komplexitätsfaktoren}
Für eine detaillierte Simulation werden die Bearbeitungszeiten basierend auf dem Dokumenttyp skaliert. Die Faktoren beruhen auf Annahmen.
\\\\
\textbf{Berechnung der Dauer für \glqq Review Documents\grqq{}:} \\
Die Dauer dieser Aktivität berechnet sich dynamisch anhand der Summe der Komplexitätsfaktoren aller zu prüfenden Dokumente (Batch-Verarbeitung):
\[
\text{Dauer} = \text{Basisdauer} (\approx 5\,\text{min}) \times \sum (\text{Komplexitätsfaktoren})
\]
(Siehe Logik in \texttt{sim\_ocel.ipynb}).

\begin{table}[H]
\centering
\small
\setlength{\tabcolsep}{4pt}
\begin{tabular}{@{} >{\raggedright\arraybackslash}p{0.38\textwidth}
       >{\raggedright\arraybackslash}p{0.30\textwidth}
       >{\raggedright\arraybackslash}p{0.22\textwidth} @{}}
\toprule
Dokument & Bedingung & Komplexitäts\-faktor \\
\midrule
Formblatt 1 (Antrag) & immer & 1.0 \\[2pt]
Immatrikulations\-bescheinigung & immer & 0.5 \\[2pt]
Formblatt 3 (Einkommen Eltern) & falls {\ttfamily Application.\allowbreak is\_parent\_\allowbreak independent = FALSE} & 1.5 \\[2pt]
Einkommens\-nachweise (Eltern) & falls Formblatt 3 vorhanden (i.d.R. 2 Stück) & 1.3 \\[2pt]
Mietbescheinigung (Wohnnachweis) & falls {\ttfamily Application.\allowbreak housing\_type \(\neq\) 'Eltern'} & 0.8 \\
\bottomrule
\end{tabular}
\vspace{1ex}
\caption{Dokumenttypen und Komplexitätsfaktoren}
\label{tab:doctypes}
\normalsize
\end{table}



\section{Object-Centric Data \& Logic}
Das Simulationsmodell basiert auf einem objekt-zentrierten Ansatz (Object-Centric Process Mining), der über den klassischen Event-Log-Standard hinausgeht.

\subsection{Objekt-Typen}
\begin{itemize}
    \item \textbf{Application:} Das zentrale Case-Objekt, das den gesamten Antragsprozess bündelt. Attribute: \texttt{application\_id}, \texttt{student\_id}, \texttt{is\_parent\_independent}, \texttt{housing\_type}, \texttt{status}, \texttt{deviation\_type}.
    \item \textbf{Document:} Sekundäre Objekte mit eigenem Lebenszyklus, die mit dem Antrag verknüpft sind. Attribute: \texttt{document\_id}, \texttt{doc\_type}, \texttt{status} (Missing/Received), \texttt{submission\_time}.
    \item \textbf{Event:} Verknüpft Aktivitäten mit einem oder mehreren Objekten. Attribute: \texttt{activity}, \texttt{timestamp}, \texttt{org\_resource}, \texttt{linked\_objects}.
\end{itemize}

\subsection{Dokumenten-Generierung und Lebenszyklus}
Vor der Aktivität \glqq Application started\grqq{} werden technisch bereits alle potenziellen Dokumente (z.\,B. Formblatt 1, Immatrikulationsbescheinigung) instanziiert. Auf Application started werden alle Dokumente, die in der Config mit \texttt{system_mandatory} markiert sind, auf \texttt{Received} gesetzt. Außerdem wird auf Basis der Gateway-Wahrscheinlichkeiten der Typ der benötigten Dokumente bestimmt und darüberhinaus ein mit einer Wahrscheinlichkeit von \texttt{40\%} als \texttt{Missing} markiert.
\\
\begin{itemize}
    \item \textbf{Initialer Status:} Alle Dokumente starten im Status \texttt{Missing}.
    \item \textbf{Status-Wechsel:} Der Status ändert sich erst auf \texttt{Received}, wenn die entsprechende Aktivität (z.\,B. \glqq Application started\grqq{}, \glqq Receive Application\grqq{}, \glqq Receive Parent Data\grqq{}, \glqq Review Documents\grqq{}) tatsächlich ausgeführt wird.
\end{itemize}

Dies ermöglicht die getrennte Nachverfolgung jedes einzelnen Belegs (Unvollständigkeit) parallel zum Hauptantrag.

Darüberhinaus gibt es in Review Documents eine Parameter \texttt{p_invalid} mit einer Wahrscheinlichkeit von \texttt{10\%}, dass ein Dokument als \texttt{Missing} markiert wird, um unvollständige Dokumente zu simulieren. Sobald diese erfolgreich reviewed sind, werden diese Dokumente keiner Prüfung mehr unterzogen.

\subsection{Object-to-Event Mapping}
Um den Lebenszyklus des Antrags und der einzelnen Dokumente getrennt voneinander, aber synchronisiert abzubilden, wird folgendes Mapping-Schema angewendet:

\begin{itemize}
    \item \textbf{Leading Object:} Das Objekt \texttt{Application} wird mit \textbf{allen} Aktivitäten verknüpft, um den durchgängigen Prozessfluss zu gewährleisten.
    \item \textbf{Secondary Object:} Das Objekt \texttt{Document} wird \textbf{exklusiv} mit Aktivitäten verknüpft, die eine physische Bearbeitung oder Zustandsänderung eines Dokuments darstellen (z.\,B. \textit{Request, Receive, Review}).
\end{itemize}

Dies ermöglicht im Process Mining die Analyse von 1:n-Beziehungen (ein Antrag hat \(n\) Dokumente, die sich unterschiedlich verhalten).
\\
\small
\setlength{\tabcolsep}{4pt}
\begin{longtable}{@{} >{\raggedright\arraybackslash}p{0.30\textwidth}
       >{\raggedright\arraybackslash}p{0.30\textwidth}
       >{\raggedright\arraybackslash}p{0.30\textwidth} @{}}
\toprule
Aktivität & Verknüpfte Objekte & Begründung \\
\midrule
\endhead
Application started & Application & Initialisierung des Cases \\
Request Parent Data & Application, \textbf{Document} & Erzeugung/Anforderung der Eltern-Objekte \\
Receive Parent Data & Application, \textbf{Document} & Eingang der Eltern-Unterlagen \\
Send Application Mail & Application & System-Notifikation an das BAföG-Amt Hamburg über Antragseingang \\
Receive Application & Application, \textbf{Document} & Eingang der Basis-Dokumente \\
Review Documents & Application, \textbf{Document} & Inhaltliche Prüfung pro Einzel-Dokument \\
Request Missing Documents & Application, \textbf{Document} & Nachforderung fehlender Dokumente \\
Receive Missing Documents & Application, \textbf{Document} & Eingang der nachgereichten Dokumente \\
Assess Application & Application & Prüfung des Gesamtantrags \\
Calculate Claim & Application & Berechnung des BAföG-Satzes \\
Send Notification & Application & Positiver Bescheid (Bewilligung) \\
Send Rejection & Application & Negativer Bescheid (Ablehnung) \\
Archive Documents & Application, \textbf{Document} & Archivierung aller Unterlagen \\
Print Documents & Application, \textbf{Document} & Drucken aller Unterlagen \\
Application handled & Application & Technischer Abschluss des Cases \\
\bottomrule
\\[1ex]
\caption{Object-to-Event Mapping} \label{tab:mapping} \\
\end{longtable}
\normalsize

\subsection{Synchronisations-Logik und Dokumentenkategorien}
Um die Modellkomplexität zu reduzieren, werden Elternunterlagen nicht in individuelle Rollen (Mutter/Vater) aufgeteilt, sondern als ein aggregiertes Nachweisobjekt behandelt. Die OCPM-Logik greift hier durch die Synchronisation unterschiedlicher Dokumentenkategorien:

\begin{enumerate}
    \item \textbf{Differenzierte Wartezeiten:} Das Dokument der Kategorie \glqq Parent\grqq{} (z.\,B. Einkommensnachweis) erhält in der Simulation eine signifikant längere Wartezeit (Exponentialverteilung, \(\mu \approx 7\) Tage) als Dokumente der Kategorie \glqq Student\grqq{} (\(\mu \approx 2\) Tage).
    \item \textbf{Synchronisation am Gateway:} Der Prozessschritt \glqq Assess Application\grqq{} fungiert als Synchronisationspunkt. Er darf erst starten, wenn \textbf{alle} verknüpften Dokumente (sowohl Student als auch Eltern) den Status \glqq Received\grqq{} erreicht haben.
\end{enumerate}

Dies demonstriert den Core-Benefit von Object-Centric Process Mining: Die Analyse zeigt, dass der Gesamtprozess oft durch ein einzelnes, komplexes Objekt (hier: Elternnachweis) blockiert wird, während andere Objekte (z.\,B. Immatrikulationsbescheinigung) bereits vorliegen (Wartezeiten-Paradoxon).

\section{Ressourcen}
\subsection{System}
\begin{itemize}
  \item 24/7 verfügbar, Kapazität hoch (9999)
  \item Verfügbarkeit: 00:00--23:59, alle Tage.
\end{itemize}

\subsection{Clerk (Sachbearbeitung)}
\begin{itemize}
  \item Anzahl: Inlandsförderung mit 10 Sachbearbeitern, in Realität 32\cite{bewilligung2012}, jedoch 10 um Warteschlangen zu simulieren.
  \item Verfügbarkeit: Mo--Fr 07:30--16:00.
  \item \textbf{Queuing:} Aufgaben, die außerhalb der Dienstzeiten eintreffen (z.\,B. durch Online-Antragstellung am Abend), werden in einer Warteschlange (Backlog) gesammelt und erst mit Beginn der nächsten Schicht (07:30 Uhr) bearbeitet. Dies erklärt Wartezeiten trotz rechnerisch ausreichender Kapazität.
\end{itemize}

\section{Geplante Deviations}
Um einen Conformance Check zu demonstrieren, werden gezielt Abweichungen in den Simulationsdaten erzeugt, die vom Standardprozess (Happy Path) abweichen.

\subsection{Switched Activities}
\textbf{Szenario:} Bei \textbf{1\,\%} der Fälle wird der Schritt \glqq Assess Application\grqq{} mit \glqq Calculate Claim\grqq{} in der Reihenfolge vertauscht (Simulation Parameter \texttt{activity\_switch}). \\\\
\textbf{Bedeutung:} Dies simuliert eine Abweichung der Prozessreihenfolge. Im Conformance Check muss dies als \textbf{Sequence Violation} erkannt werden.

\subsection{Direkte Ablehnung (Shortened Path)}
\textbf{Szenario:} Bei \textbf{3\,\%} der Fälle bricht der Prozess nach \glqq Review Documents\grqq{} sofort mit einer Ablehnung ab (Simulation Parameter \texttt{direct\_rejection}). \\\\
\textbf{Bedeutung:} Simuliert sofortige Ablehnung (z.\,B. bei offensichtlicher Unzuständigkeit). Erkennbar als \textbf{Skipped Activities}.

\subsection{Blind Approval (Compliance Violation)}
\textbf{Szenario:} Bei \textbf{2\,\%} wird der Antrag ohne Prüfung (\glqq Assess\grqq{}) direkt genehmigt (Simulation Parameter \texttt{blind\_approval}). \\\\
\textbf{Bedeutung:} Ein schwerwiegender Compliance-Verstoß (Genehmigung ohne Prüfung).

\subsection{document\_invalid (Rework)}
\textbf{Szenario:} Bei \textbf{20\,\%} der Fälle wird ein Dokument im Review als ungültig oder fehlerhaft abgelehnt (Simulation Parameter \texttt{document\_invalid}). \\\\
\textbf{Bedeutung:} Dies führt zu einer weiteren Review-Runde und erhöht die Durchlaufzeit massiv.

\subsection{Ablehnung wegen Fristablauf (Timeout)}
\textbf{Szenario:} Bei \textbf{4\,\%} springt der Prozess nach einer Wartezeit von >30 Tagen direkt zur Ablehnung. \\\\
\textbf{Bedeutung:} Automatische Ablehnung bei Inaktivität.

\subsection{Unmapped Activities (Model Moves)}
\textbf{Szenario:} Die Aktivitäten \glqq Print Documents\grqq{} und \glqq Archive Documents\grqq{} werden von Sachbearbeitern (Clerks) ausgeführt, sind aber \textbf{nicht im Soll-Prozess (BPMN) enthalten}. \\\\
\textbf{Bedeutung:} Diese Schritte tauchen als \textbf{Log Moves} (im Log, aber nicht im Modell) im Conformance Checking auf und müssen gefiltert oder als Abweichung akzeptiert werden (Annahme: Medienbrüche in der Realität).

\section{Simulationsempfehlungen}
\begin{itemize}
  \item KPIs: Cycle Time je Antrag, Auslastung Sachbearbeiter, Wartezeiten pro Queue, Bewilligungsquote.
  \item Experimente: Sensitivität auf Dokumentvollständigkeit (Gateway), Variation Receive-Document-Mittelwert, Personalkapazität (z.B. 10--16 FTE), saisonale Peaks vs. Normalsemester.
\end{itemize}

\section{Limitations}
Die Simulation unterliegt folgenden Einschränkungen:

\begin{itemize}
    \item \textbf{Fokus auf Erstanträge:} Die Simulation bildet ausschließlich Erstanträge ab. Folgeanträge, Schüler-BAföG und Auslands-BAföG werden nicht berücksichtigt.
    
    \item \textbf{Ausschließlich digitale Anträge:} Es werden nur Anträge über BAföG-Digital simuliert. Papieranträge und hybride Einreichungskanäle sind ausgeschlossen.
    
    \item \textbf{Aggregierte Elternunterlagen:} Elternunterlagen (Einkommen Mutter/Vater) werden nicht individuell unterschieden, sondern als aggregiertes Nachweisobjekt behandelt.
    
    \item \textbf{Vereinfachte Dokumentenlogik:} Die Wahrscheinlichkeit, dass ein Dokument initial fehlt, ist fest auf 40\,\% gesetzt. In der Realität variiert dies je nach Dokumenttyp und Antragstellerprofil.
    
    \item \textbf{Keine Ressourcenausfälle:} Krankheitsausfälle, Urlaub oder Schulungen der Sachbearbeiter werden nicht modelliert. Die Kapazität bleibt konstant bei 10 FTE.
    
    \item \textbf{Feste Review-Runden:} Die maximale Anzahl der Review-Runden ist auf 10 begrenzt (\texttt{max\_review\_rounds}). Endlose Nachforderungsschleifen werden verhindert.
    
    \item \textbf{Keine saisonalen Variationen innerhalb des Semesters:} Die Interarrival-Raten sind über den gesamten Simulationszeitraum exponential verteilt und bilden keine intra-saisonalen Schwankungen (z.\,B. Prüfungsphasen) ab.
    
    \item \textbf{Keine Bewerbungszeitfenster 00:00--08:00 Uhr wochentags:} In der aktuellen Konfiguration fehlt ein explizites Interarrival-Zeitfenster für die Nacht (00:00--08:00 Uhr) an Werktagen; diese Zeiträume werden mit einem Default-Wert behandelt.
    
    \item \textbf{Statische Ablehnungsquote:} Die Bewilligungsquote von 84\,\% ist konstant und berücksichtigt keine Variation nach Antragstellerprofil oder Dokumentqualität.
\end{itemize}

\newpage

\bibliographystyle{apalike}
\begin{thebibliography}{9}

\bibitem[Bundesrechnungshof, 2024]{brh2024}
Bundesrechnungshof (2024):
\textit{Bericht zum BAföG – Volltext}.
Textziffern (Tz.) 2.2.1, 3.2, 9.1, 9.2 u.\,a.
URL: \url{https://www.bundesrechnungshof.de/SharedDocs/Downloads/DE/Berichte/2024/bafoeg-volltext.pdf?__blob=publicationFile&v=2}

\bibitem[Bundesregierung, 2010]{einfacher2010}
Bundesregierung; Nationale Normenkontrollrat (2010):
\textit{Einfacher zum Studierenden-BAföG – Abschlussbericht}.
URL: \url{https://www.bundesregierung.de/resource/blob/2065474/396444/d606b826f7415afb0c091c30e968e383/2010-03-17-abschlussbericht-einfacher-zum-bafoeg-data.pdf?download=1}

\bibitem[Bundesregierung, 2012]{bewilligung2012}
Bundesregierung (2012):
\textit{Projektbericht zur BAföG-Verwaltung – Bewilligungsquoten Inland}.
Tabellen 22 und 23.
URL: \url{https://www.bundesregierung.de/resource/blob/974430/455514/00c68b52fe5801d0f59643345aa942a2/2012-06-22-projektbericht-7-data.pdf?download=1}

\bibitem[Studentenwerk SH, 2024]{sh2024}
Studentenwerk Schleswig-Holstein (2024):
\textit{Warum dauert die BAföG-Bearbeitung länger? – FAQ}.
URL: \url{https://studentenwerk.sh/de/bafoeg-warum-die-antragstellung-laenger-dauert}

\bibitem[Destatis, 2024]{destatis2024}
Statistisches Bundesamt (Destatis) (2024):
\textit{BAföG-Geförderte nach Ländern}.
GENESIS-Tabelle 21411-0020.
URL: \url{https://www-genesis.destatis.de/datenbank/online/statistic/21411/table/21411-0020}

\bibitem[Statistik Bayern, 2022]{statbay2022}
Bayerisches Landesamt für Statistik (2022):
\textit{BAföG-Geförderte – Anteil elternunabhängiger Förderung}.
Statistische Berichte, Kennziffer K9100C.
URL: \url{https://www.statistik.bayern.de/mam/produkte/veroffentlichungen/statistische_berichte/k9100c_202200.pdf}

\end{thebibliography}


\end{document}