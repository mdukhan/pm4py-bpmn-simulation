\documentclass[a4paper,11pt]{article}
\usepackage[margin=1in]{geometry}
\usepackage[utf8]{inputenc}
\usepackage[T1]{fontenc}
\usepackage[english]{babel}
\usepackage{booktabs}
\usepackage{array}
\usepackage{hyperref}
\usepackage{longtable}
\usepackage{amsmath}
\usepackage{apalike}
\usepackage{graphicx} 
\usepackage{rotating}
\usepackage{float}

\title{Parametrization for the BAföG OCEL Simulation \\[1ex] \large Business Process Management (WS 25/26) \\ Group 10}
\author{Alina Thoden \\ Haya Edris \\ Mansour Dukhan \\ David Derr \\ Büsra Bugrahan \\ Jan Gödicke}
\date{\today}

\begin{document}
\maketitle

\section{Process Context}

\subsection{Scenario and Data Basis}
The subject of the simulation is the influx of BAföG applications for Hamburg for the winter semester 2024/2025. The peak load is characterized by the start of the semester, with about 63\,\% of applications arriving around the start of the winter term~\cite{sh2024}. The data basis for Hamburg is 15,564 funded students in the year 2024~\cite{destatis2024}.

\subsection{Process Focus}
The focus is on the processing of initial applications that are submitted exclusively digitally via BAföG-Digital. The scenario thus considers only applications via BAföG-Digital and excludes paper or mixed channels as well as follow-up applications, student and study-abroad BAföG. A separation between domestic and study-abroad BAföG is possible, as personnel capacity is divided between them~\cite{bewilligung2012}.
\\\\
The start point is ``Application started'', i.e., the time of data entry into the system. Preparatory work by the student is not considered, as it does not influence the office throughput time.

\subsection{Volume Assumptions}
We simulate a High-Load Scenario, where the entire application volume (9,800 at the start of the winter semester) is treated as initial applications to test the system's resilience under maximum complexity.

\subsection{Modeling Approach}
The current BPMN \autoref{fig:bafoeg_ocel} is more data-driven than the initial version \autoref{fig:bafoeg_v1}. The current version concentrates on activities that would cause changes in the databases, such as CRUD operations.

\begin{sidewaysfigure}
    \centering
    \includegraphics[width=\textwidth]{bafoeg_v1.png}
    \caption{Initial Version: BPMN of the BAföG Process}
    \label{fig:bafoeg_v1}
\end{sidewaysfigure}

\begin{sidewaysfigure}
    \centering
    \includegraphics[width=\textwidth]{bafoeg_ocel.png}
    \caption{Current BPMN of the BAföG Process}
    \label{fig:bafoeg_ocel}
\end{sidewaysfigure}

\subsection{Scenario Setup \& Output}
\begin{itemize}
    \item \textbf{Timeframe:} The simulation starts on \textbf{15.09.2024} and runs for a period of \textbf{80 days} to cover the entire winter semester including follow-up time.
    \item \textbf{Reproducibility:} To ensure reproducible results, a fixed \textbf{Random Seed (42)} is used.
    \item \textbf{Output:} The simulation generates relational \textbf{CSV tables} (Events, Applications, Documents, Object-Links) that correspond in their structure to an Object-Centric Event Log (OCEL). This allows for direct analysis in process mining tools that support object-centric data.
\end{itemize}

\section{Interarrival}
The load peaks are based on the start of the winter semester. According to ~\cite{sh2024}, about 63\,\% of applications are received around the start of the winter term. Applied to Hamburg, this corresponds to about 9,800 of 15,564 funded students~\cite{destatis2024}. We assume the following interarrival times:

\begin{longtable}{@{}p{2.5cm}p{2cm}p{2.5cm}p{3cm}p{5cm}@{}}
\toprule
Time Window & Weekdays & Distribution & Parameter & Assumption \\
\midrule
08:00--16:00 & Mon--Fri & Exponential & Mean 1.5 min & Peak: Main business hours (High Load Scenario) \\
16:00--21:00 & Mon--Fri & Exponential & Mean 2.7 min & Peak: After lectures (High Load Scenario) \\
21:00--24:00 & Mon--Fri & Exponential & Mean 8.1 min & Late submission, moderate density \\
All day & Sat--Sun & Exponential & Mean 10.8 min & Weekend, reduced rate \\
\bottomrule
\\[1ex]
\caption{Interarrival times for the simulation} \label{tab:interarrival} \\
\end{longtable}

\section{Gateways}
\subsection{Parent Data Required?}
According to Statistics Bavaria 2022~\cite{statbay2022}, about 18\% of funded students are parent-independent (13,637 of 74,771). From this we derive that in 80\% of cases parent documents are required, while 20\% are funded independently of parents. This controls the attribute \texttt{Application.is\_parent\_independent} (FALSE = Parent documents required).

\subsection{Documents Missing?} 
According to technical literature, only 1--2\% of paper initial applications are complete~\cite{einfacher2010}, while for continued funding about 35\% are submitted completely~\cite{sh2024}. Since the focus is exclusively on initial applications, the risk of incomplete documents is higher (approx. 98\% for paper). Through the use of the digital assistant BAföG Digital, an improvement is assumed, which is why we set the risk ``Documents Missing'' to 40\% and ``Complete'' to 60\%. This gateway is controlled by the attribute \texttt{Document.status} (``Missing'' vs. ``Received'').

\subsection{Eligibility Decision?}
According to~\cite{bewilligung2012}, the share of unapproved applications is 16\%. The path probabilities are thus 84\% for approved and 16\% for rejected. This gateway is controlled by the attribute \texttt{Application.status} (``Approved'' vs. ``Rejected'').

\section{Activity Durations}
To derive the activity durations, the standard times from Table 22 (``Average standard times of application processing in the Student Union Hamburg for domestic funding'')~\cite{bewilligung2012} were used. For the initial application scenario, this results in a total of 83 minutes. The shares of the individual activities were determined consistently with the standard times given in Table 22 and distributed to the activity groups listed below:

  \begin{itemize}
      \item \textbf{Receive Application:} corresponds to creation of the paper file \(\approx 13\,\text{min}\)
      \item \textbf{Review:} corresponds to completeness check \(\approx 13\,\text{min}\).
      \item \textbf{Request Missing Data}  corresponds to obtaining missing data or information \(\approx 12\,\text{min}\).
      \item \textbf{Assess:} corresponds to half of the time of performing calculations/assessments \(\approx 15\,\text{min}\).
      \item \textbf{Calculate:} corresponds to half of the time of performing calculations/assessments incl. checking and correcting results \(\approx 20\,\text{min}\).
      \item \textbf{Notification/Rejection:} corresponds to preparation, sending \(\approx 10\,\text{min}\).
  \end{itemize}

The activity durations of the casework were transferred into stochastic distributions for parameterization. Manual tasks are modeled by normal distributions (with plausible standard deviations \(\sigma\) as well as specified minimum and maximum values), system-side activities by uniform distributions and waiting times by exponential distributions.
\\
\small
\setlength{\LTpre}{0pt}
\setlength{\LTpost}{0pt}
\begin{longtable}{@{} >{\raggedright\arraybackslash}p{0.18\textwidth}
          >{\raggedright\arraybackslash}p{0.12\textwidth}
          >{\raggedright\arraybackslash}p{0.15\textwidth}
          >{\raggedright\arraybackslash}p{0.22\textwidth}
          >{\raggedright\arraybackslash}p{0.33\textwidth} @{}}
\toprule
Activity & Resource & Distribution & Parameter & Justification \\
\midrule
\endhead
Application started & System & - & 0 min & Start-Event (Data Entry) \\
Request Parent Data & System & Uniform & 0.5--2 min & Automated mail to parents \\
Receive Parent Data & External Actor & Normal & $\mu=10080$, $\sigma=3600$ & Waiting time (Assumption: Normal distribution) \\
Send Application Mail & System & Uniform & 1--3 min & Automatic generation + sending \\
Receive Application & Clerk & Normal & $\mu=13$, $\sigma=4$ & Creation of paper file \\
Review Documents & Clerk & Normal & $\mu=5$, $\sigma=2$ & Time per document (Assumption sim\_config) \\
Request Missing Documents & Clerk & Normal & $\mu=12$, $\sigma=3$ & Obtain missing data \\
Receive Missing Documents & External Actor & Normal & $\mu=10080$, $\sigma=2880$ & Waiting time (Assumption: Normal distribution) \\
Print Documents & Clerk & Uniform & 2--5 min & Media disruption: Batch printing (Assumption: Deviation) \\
Assess Application & Clerk & Normal & $\mu=15$, $\sigma=4$ & Assessment \\
Calculate Claim & Clerk & Normal & $\mu=20$, $\sigma=5$ & Calculation \\
Send Notification & Clerk & Normal & $\mu=10$, $\sigma=3$ & Sending notification \\
Send Rejection & Clerk & Normal & $\mu=10$, $\sigma=3$ & Sending rejection \\
Archive Documents & Clerk & Uniform & 2--5 min & Archiving (Assumption: Deviation) \\
Application handled & System & - & 0 min & End-Event (Technical completion) \\
\bottomrule
\\[1ex]
\caption{Distribution of activity durations} \label{tab:activities} \\
\end{longtable}
\normalsize

\subsection{Document Types and Complexity Factors}
For a detailed simulation, the processing times are scaled based on the document type. The factors are based on assumptions.
\\\\
\textbf{Calculation of duration for ``Review Documents'':} \\
The duration of this activity is calculated dynamically based on the sum of the complexity factors of all documents to be checked (batch processing):
\[
\text{Duration} = \text{Base Duration} (\approx 5\,\text{min}) \times \sum (\text{Complexity Factors})
\]
(See logic in \texttt{sim\_ocel.ipynb}).

\begin{table}[H]
\centering
\small
\setlength{\tabcolsep}{4pt}
\begin{tabular}{@{} >{\raggedright\arraybackslash}p{0.38\textwidth}
       >{\raggedright\arraybackslash}p{0.30\textwidth}
       >{\raggedright\arraybackslash}p{0.22\textwidth} @{}}
\toprule
Document & Condition & Complexity Factor \\
\midrule
Formblatt 1 (Application) & always & 1.0 \\[2pt]
Enrollment Certificate & always & 0.5 \\[2pt]
Formblatt 3 (Parent Income) & if {\ttfamily Application.\allowbreak is\_parent\_\allowbreak independent = FALSE} & 1.5 \\[2pt]
Income Proof (Parents) & if Formblatt 3 present (usually 2 pieces) & 1.3 \\[2pt]
Rent Certificate & if {\ttfamily Application.\allowbreak housing\_type \(\neq\) 'Eltern'} & 0.8 \\
\bottomrule
\end{tabular}
\vspace{1ex}
\caption{Document types and complexity factors}
\label{tab:doctypes}
\normalsize
\end{table}



\section{Object-Centric Data \& Logic}
The simulation model is based on an object-centric approach (Object-Centric Process Mining), which goes beyond the classic Event Log standard.

\subsection{Object Types}
\begin{itemize}
    \item \textbf{Application:} The central Case object that bundles the entire application process. Attributes: \texttt{application\_id}, \texttt{student\_id}, \texttt{is\_parent\_independent}, \texttt{housing\_type}, \texttt{status}, \texttt{deviation\_type}.
    \item \textbf{Document:} Secondary objects with their own lifecycle, which are linked to the application. Attributes: \texttt{document\_id}, \texttt{doc\_type}, \texttt{status} (Missing/Received), \texttt{submission\_time}.
    \item \textbf{Event:} Links activities with one or more objects. Attributes: \texttt{activity}, \texttt{timestamp}, \texttt{org\_resource}, \texttt{linked\_objects}.
\end{itemize}

\subsection{Document Generation and Lifecycle}
Before the activity ``Application started'', all potential documents (e.g. Formblatt 1, Enrollment Certificate) are technically already instantiated. On Application started, all documents marked with \texttt{system\_mandatory} in the config are set to \texttt{Received}. Furthermore, based on the gateway probabilities, the type of required documents is determined and additionally marked as \texttt{Missing} with a probability of \texttt{40\%}.
\\
\begin{itemize}
    \item \textbf{Initial Status:} All documents start in the status \texttt{Missing}.
    \item \textbf{Status Change:} The status only changes to \texttt{Received} when the corresponding activity (e.g. ``Application started'', ``Receive Application'', ``Receive Parent Data'', ``Review Documents'') is actually executed.
\end{itemize}

This allows for the separate tracking of each individual document (incompleteness) parallel to the main application.

Furthermore, in Review Documents there is a parameter \texttt{p\_invalid} with a probability of \texttt{10\%} that a document is marked as \texttt{Missing} to simulate incomplete documents. As soon as these are successfully reviewed, these documents are no longer subjected to any check.

\subsection{Object-to-Event Mapping}
To depict the lifecycle of the application and the individual documents separately from each other, but synchronized, the following mapping scheme is applied:

\begin{itemize}
    \item \textbf{Leading Object:} The object \texttt{Application} is linked with \textbf{all} activities to ensure the continuous process flow.
    \item \textbf{Secondary Object:} The object \texttt{Document} is linked \textbf{exclusively} with activities that represent a physical processing or state change of a document (e.g. \textit{Request, Receive, Review}).
\end{itemize}

This allows the analysis of 1:n relationships in Process Mining (one application has \(n\) documents that behave differently).
\\
\small
\setlength{\tabcolsep}{4pt}
\begin{longtable}{@{} >{\raggedright\arraybackslash}p{0.30\textwidth}
       >{\raggedright\arraybackslash}p{0.30\textwidth}
       >{\raggedright\arraybackslash}p{0.30\textwidth} @{}}
\toprule
Activity & Linked Objects & Justification \\
\midrule
\endhead
Application started & Application & Initialization of the Case \\
Request Parent Data & Application, \textbf{Document} & Generation/Request of Parent Objects \\
Receive Parent Data & Application, \textbf{Document} & Receipt of Parent Documents \\
Send Application Mail & Application & System notification to the BAföG office Hamburg about application receipt \\
Receive Application & Application, \textbf{Document} & Receipt of Basic Documents \\
Review Documents & Application, \textbf{Document} & Content check per single document \\
Request Missing Documents & Application, \textbf{Document} & Requesting missing documents \\
Receive Missing Documents & Application, \textbf{Document} & Receipt of submitted documents \\
Assess Application & Application & Check of the entire application \\
Calculate Claim & Application & Calculation of the BAföG rate \\
Send Notification & Application & Positive notification (Approval) \\
Send Rejection & Application & Negative notification (Rejection) \\
Archive Documents & Application, \textbf{Document} & Archiving of all documents \\
Print Documents & Application, \textbf{Document} & Printing of all documents \\
Application handled & Application & Technical completion of the Case \\
\bottomrule
\\[1ex]
\caption{Object-to-Event Mapping} \label{tab:mapping} \\
\end{longtable}
\normalsize

\subsection{Synchronization Logic and Document Categories}
To reduce model complexity, parent documents are not split into individual roles (mother/father), but treated as an aggregated proof object. The OCPM logic applies here through the synchronization of different document categories:

\begin{enumerate}
    \item \textbf{Differentiated waiting times:} The document of the category ``Parent'' (e.g. Income Proof) receives a significantly longer waiting time in the simulation (exponential distribution, \(\mu \approx 7\) days) than documents of the category ``Student'' (\(\mu \approx 2\) days).
    \item \textbf{Synchronization at the Gateway:} The process step ``Assess Application'' acts as a synchronization point. It may only start when \textbf{all} linked documents (both Student and Parent) have reached the status ``Received''.
\end{enumerate}

This demonstrates the Core Benefit of Object-Centric Process Mining: The analysis shows that the overall process is often blocked by a single, complex object (here: Parent Proof), while other objects (e.g. Enrollment Certificate) are already present (Waiting Time Paradox).

\section{Resources}
\subsection{System}
\begin{itemize}
  \item Available 24/7, high capacity (9999)
  \item Availability: 00:00--23:59, all days.
\end{itemize}

\subsection{Clerk (Caseworker)}
\begin{itemize}
  \item Number: Domestic funding with 10 clerks, in reality 32\cite{bewilligung2012}, however 10 to simulate queues.
  \item Availability: Mon--Fri 07:30--16:00.
  \item \textbf{Queuing:} Tasks that arrive outside service hours (e.g. through online application in the evening) are collected in a queue (backlog) and only processed at the beginning of the next shift (07:30 am). This explains waiting times despite calculably sufficient capacity.
\end{itemize}

\section{Planned Deviations}
To demonstrate a Conformance Check, deviations are deliberately generated in the simulation data that deviate from the standard process (Happy Path).

\subsection{Switched Activities}
\textbf{Scenario:} In \textbf{1\,\%} of cases, the step ``Assess Application'' is swapped with ``Calculate Claim'' in the sequence (Simulation Parameter \texttt{activity\_switch}). \\\\
\textbf{Meaning:} This simulates a deviation in the process sequence. In the Conformance Check, this must be detected as a \textbf{Sequence Violation}.

\subsection{Direct Rejection (Shortened Path)}
\textbf{Scenario:} In \textbf{3\,\%} of cases, the process aborts immediately with a rejection after ``Review Documents'' (Simulation Parameter \texttt{direct\_rejection}). \\\\
\textbf{Meaning:} Simulates immediate rejection (e.g. in case of obvious lack of jurisdiction). Recognizable as \textbf{Skipped Activities}.

\subsection{Blind Approval (Compliance Violation)}
\textbf{Scenario:} In \textbf{2\,\%}, the application is approved directly without check (``Assess'') (Simulation Parameter \texttt{blind\_approval}). \\\\
\textbf{Meaning:} A serious compliance violation (approval without check).

\subsection{document\_invalid (Rework)}
\textbf{Scenario:} In \textbf{20\,\%} of cases, a document is rejected as invalid or incorrect during the review (Simulation Parameter \texttt{document\_invalid}). \\\\
\textbf{Meaning:} This leads to a further review round and increases the cycle time massively.

\subsection{Rejection due to Timeout}
\textbf{Scenario:} In \textbf{4\,\%}, the process jumps directly to rejection after a waiting time of >30 days. \\\\
\textbf{Meaning:} Automatic rejection in case of inactivity.

\subsection{Unmapped Activities (Model Moves)}
\textbf{Scenario:} The activities ``Print Documents'' and ``Archive Documents'' are performed by clerks, but are \textbf{not included in the target process (BPMN)}. \\\\
\textbf{Meaning:} These steps appear as \textbf{Log Moves} (in the log, but not in the model) in Conformance Checking and must be filtered or accepted as a deviation (Assumption: Media disruptions in reality).

\section{Simulation Recommendations}
\begin{itemize}
  \item KPIs: Cycle Time per application, utilization of clerks, waiting times per queue, approval rate.
  \item Experiments: Sensitivity to document completeness (Gateway), variation Receive-Document mean, personnel capacity (e.g. 10--16 FTE), seasonal peaks vs. normal semester.
\end{itemize}

\section{Limitations}
The simulation is subject to the following limitations:

\begin{itemize}
    \item \textbf{Focus on initial applications:} The simulation depicts exclusively initial applications. Follow-up applications, student BAföG and study-abroad BAföG are not considered.
    
    \item \textbf{Exclusively digital applications:} Only applications via BAföG-Digital are simulated. Paper applications and hybrid submission channels are excluded.
    
    \item \textbf{Aggregated parent documents:} Parent documents (Income Mother/Father) are not distinguished individually, but treated as an aggregated proof object.
    
    \item \textbf{Simplified document logic:} The probability that a document is initially missing is fixed at 40\,\%. In reality, this varies depending on document type and applicant profile.
    
    \item \textbf{No resource outages:} Sickness, leave or training of clerks are not modeled. Capacity remains constant at 10 FTE.
    
    \item \textbf{Fixed review rounds:} The maximum number of review rounds is limited to 10 (\texttt{max\_review\_rounds}). Endless request loops are prevented.
    
    \item \textbf{No seasonal variations within the semester:} The interarrival rates are exponentially distributed over the entire simulation period and do not depict intra-seasonal fluctuations (e.g. exam phases).
    
    \item \textbf{No application time windows 00:00--08:00 weekdays:} In the current configuration, an explicit interarrival time window for the night (00:00--08:00 am) on weekdays is missing; these periods are treated with a default value.
    
    \item \textbf{Static rejection rate:} The approval rate of 84\,\% is constant and does not take into account variation by applicant profile or document quality.
\end{itemize}

\newpage

\bibliographystyle{apalike}
\begin{thebibliography}{9}

\bibitem[Bundesrechnungshof, 2024]{brh2024}
Bundesrechnungshof (2024):
\textit{Bericht zum BAföG – Volltext}.
Textziffern (Tz.) 2.2.1, 3.2, 9.1, 9.2 u.\,a.
URL: \url{https://www.bundesrechnungshof.de/SharedDocs/Downloads/DE/Berichte/2024/bafoeg-volltext.pdf?__blob=publicationFile&v=2}

\bibitem[Bundesregierung, 2010]{einfacher2010}
Bundesregierung; Nationale Normenkontrollrat (2010):
\textit{Einfacher zum Studierenden-BAföG – Abschlussbericht}.
URL: \url{https://www.bundesregierung.de/resource/blob/2065474/396444/d606b826f7415afb0c091c30e968e383/2010-03-17-abschlussbericht-einfacher-zum-bafoeg-data.pdf?download=1}

\bibitem[Bundesregierung, 2012]{bewilligung2012}
Bundesregierung (2012):
\textit{Projektbericht zur BAföG-Verwaltung – Bewilligungsquoten Inland}.
Tabellen 22 und 23.
URL: \url{https://www.bundesregierung.de/resource/blob/974430/455514/00c68b52fe5801d0f59643345aa942a2/2012-06-22-projektbericht-7-data.pdf?download=1}

\bibitem[Studentenwerk SH, 2024]{sh2024}
Studentenwerk Schleswig-Holstein (2024):
\textit{Warum dauert die BAföG-Bearbeitung länger? – FAQ}.
URL: \url{https://studentenwerk.sh/de/bafoeg-warum-die-antragstellung-laenger-dauert}

\bibitem[Destatis, 2024]{destatis2024}
Statistisches Bundesamt (Destatis) (2024):
\textit{BAföG-Geförderte nach Ländern}.
GENESIS-Tabelle 21411-0020.
URL: \url{https://www-genesis.destatis.de/datenbank/online/statistic/21411/table/21411-0020}

\bibitem[Statistik Bayern, 2022]{statbay2022}
Bayerisches Landesamt für Statistik (2022):
\textit{BAföG-Geförderte – Anteil elternunabhängiger Förderung}.
Statistische Berichte, Kennziffer K9100C.
URL: \url{https://www.statistik.bayern.de/mam/produkte/veroffentlichungen/statistische_berichte/k9100c_202200.pdf}

\end{thebibliography}


\end{document}
